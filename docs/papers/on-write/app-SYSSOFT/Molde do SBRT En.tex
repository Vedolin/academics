\documentclass{sbrt2016eng}

\begin{document}

\title{How to Use Style sbrt2016eng.cls}

\author{Author 1 and Author 2
\thanks{Author 1 and Author 2�
Department of Electrical Engineering, University of SomePlace, City-PA, Brazil, E-mails: author1@ufpp.br, author2@ufp.br. This work was partially supported by CNPq (XX/XXXXX-X).} }

\maketitle

\markboth{XXXIV SIMP�SIO BRASILEIRO DE TELECOMUNICA��ES - SBrT2016, AUGUST 30 TO SEPTEMBER 02, SANTAR�M, PA} {XXXIV SIMP�SIO BRASILEIRO DE TELECOMUNICA��ES - SBrT2016, AUGUST 30 TO SEPTEMBER 02, SANTAR�M, PA}

\begin{abstract}
This article is an example of how to use a \LaTeX\  style to prepare the final or camera-ready version to appear in the Proceedings of the XXXIV Simp�sio Brasileiro de Telecomunica��es - SBrT2016.
The sbrt2016.cls style is based on the IEEEtran.cls.
\end{abstract}

\begin{keywords}
Style file, \LaTeX, SBrT2016, IEEE Conference style.
\end{keywords}

\section{Introduction}
The XXXIV Simp�sio Brasileiro de Telecomunica��es (SBrT2016) is organized by the Brazilian Telecommunications Society (SBrT).

\subsection{About the Symposium}

This yearly symposium promotes the most relevant national meeting on the area of telecommunications where important themes for the evolution of research and development of the sector are discussed.


\section{Figures and Tables}
Table \ref{tabela} is just an example \cite{ref2}.
\begin{table}[htb]
\caption{\label{tabela}\textit{Caption} comes before the table.}
\begin{center}
{\tt
\begin{tabular}{|c||c|c|c|}\hline
&title page&odd page&even page\\\hline\hline
onesided&leftTEXT&leftTEXT&leftTEXT\\\hline
twosided&leftTEXT&rightTEXT&leftTEXT\\\hline
\end{tabular}
}
\end{center}
\end{table}

Figure \ref{figura} is just an example \cite{ref2}.

\begin{figure}[hbt]
\begin{center}
\setlength{\unitlength}{0.0105in}%
\begin{picture}(242,156)(73,660)
\put( 75,660){\framebox(240,150){}} \put(105,741){\vector( 0, 1){
66}} \put(105,675){\vector( 0, 1){ 57}} \put( 96,759){\vector( 1,
0){204}} \put(105,789){\line( 1, 0){ 90}} \put(195,789){\line(
2,-1){ 90}} \put(105,711){\line( 1, 0){ 60}} \put(165,711){\line(
5,-3){ 60}} \put(225,675){\line( 1, 0){ 72}} \put(
96,714){\vector( 1, 0){204}} \put(
99,720){\makebox(0,0)[rb]{\raisebox{0pt}[0pt][0pt]{a}}}
\put(291,747){\makebox(0,0)[lb]{\raisebox{0pt}[0pt][0pt]{ o}}}
\put(291,702){\makebox(0,0)[lb]{\raisebox{0pt}[0pt][0pt]{ o}}}
\put( 99,795){\makebox(0,0)[rb]{\raisebox{0pt}[0pt][0pt]{ $M$}}}
\end{picture}
\end{center}
\caption{\label{figura}This figure us just an example. \textit{Caption} must come after the figure.}
\end{figure}

\section{Equations and Theorems}

\begin{theorem}[Name of Theorem]
Consider the system
\begin{equation}
\begin{array}{rrr}
\dot x&=&A.x+B.u\\[2mm] y&=& C.x+D.u
\end{array}
\end{equation}
\begin{equation}
\left[\begin{array}{c|c}    A & b_1 \\ \hline    c & d_1
\end{array}\right]\quad{\hbox{e}}\quad\left[\begin{array}{c|c}
A & b_2 \\ \hline    c & d_2  \end{array}\right].
\end{equation}
If $A$ is stable, then the pair $\{A,B\}$ is possibly stable and this remains for any $B$.
\end{theorem}
\begin{proof}
Demonstration is trivial and left to the interested readers.
\end{proof}

\section{Conclusions}
The paper must be in A4 size, double-column, 10pt font size and with no more than 5 pages. Abstract must have at most 100 words.

\section*{Acknowledgments}
The Technical Committee of SBrT2016 would like to thank the past symposium chairs for making this template available.

\begin{thebibliography}{99}
\bibitem{ref1} L. Lamport, \textit{A Document Preparation System: \LaTeX, User's Guide and Reference Manual}. Addison Wesley Publishing Company,
1986.
\bibitem{ref2} F. C. Silva and J. J. Sousa, ``This reference is just an example," ~\textit{Journal of Examples}, v. 5, pp. 52--55, May 1999.
\end{thebibliography}


\appendix
\section{}
Here goes some information more suitable to an appendix.


\end{document}
